\documentclass[twocolumn]{article}
\usepackage{verbatim}
\title{META II\\A SYNTAX-ORIENTED COMPILER WRITING LANGUAGE}
\date{}
\author{D. V. Schorre\\UCLA Computing Facility}
\begin{document}
\maketitle

META II is a compiler writing language which consists of syntax equations
resembling Bacus normal form and into which instructions to output
assembly language commands are inserted. Compilers have been written
in this language for VALGOL I and VALGOL II. The former is a simple
algebraic language designed for the purpose of illustrating META II.
The latter contains a fairly large subset of ALGOL 60.

The method of writing compilers which is given in detail in the paper may be
explained briefly as follows.
Each syntax equation is translated into a recusive subroutine which tests the
input string for a particular phrase structure, and deletes it if found.
Backup is avoided by the extensive use of factoring in the syntax equations.
For each source language, an interpreter is written and programs are compiled
into that interpretive language.

META II is not intended as a standard language which everyone will use to write
compilers.
Rather, it is an example of a simple working language which can give one a good
start in designing a compiler-writing compiler suited to his own needs.
Indeed, the META II compiler is written in its own language, thus lending itself
to modification.

\section{History}

The basic ideas behind META II were described in a series of three papers by
Schmidt, Metcalf, and Schorre.
These papers were presented at the 1963 National A.C.M. Convention in Denver,
and represented the activity of the Working Group on Syntax-Directed Compilers
of the Los Angeles SIGPLAN.
The methods used by that group are similar to those of Glennie and Conway, but
differ in one important respect. Both of these researchers expressed syntax in
the form of diagrams, which they subsequently coded for use on a computer.
In the case of META II, the syntax is input to the computer in a notation
resembling Backus normal form.
The method of syntax analysis discussed in this paper is entirely different
from the one used by Irons and Bastian. All of these methods can be traced back
to the mathematical study of natural languages, as described by Chomsky.

\section{Syntax Notation}

The notation used here is similar to the meta language of the ALGOL 60 report.
Probably the main difference is that this notation can be keypunched.
Symbols in the target language are represented as strings of characters,
surrounded by quotes.
Metalinguistic variables have the same form as identifiers in ALGOL, viz. a
letter followed by a sequence of letters or digits.
tems are written consecutively to indicate concatenation and separated by a
slash to indicate alternation. Each equation ends with a semicolon which,
due to keypunch liminations, is represented by a period followed by a comma.
An example of a syntax equation is:

\begin{center}
{\tt LOGICALVALUE = '.TRUE' / '.FALSE' .},
\end{center}

In the versions of ALGOL described in this paper the symbols which are usually
printed in boldface type will begin with periods, for example:

\begin{center}
{\tt .PROCEDURE .TRUE .IF }
\end{center}

To indicate that a syntactic element is optional, it may be put in alternation
with the word .EMPTY, For example:

\begin{center}
{\tt SUBSECONDARY = '*' PRIMARY / .EMPTY .}, \\
{\tt SECONDARY = PRIMARY SUBSECONDARY .},
\end{center}

By factoring, these two equations can be written as a single equation.

\begin{center}
{\tt SECONDARY = PRIMARY('*' PRIMARY / .EMPTY) .},
\end{center}

Built into the META II language is the ability to recognize three basic symbols
which are: 

\begin{enumerate}
\item Identifiers -- represented by .ID,
\item Strings -- represented by .STRING,
\item Numbers -- represented by .NUMBER.
\end{enumerate}

The definition of identifier is the same in META II as in ALGOL, viz., a letter
followed by a sequence of letters or digits.
The definition of a string is changed because of the limited character set
available on the usual keypunch.
In ALGOL, strings are surrouned by opening and closing quotation marks,
making it possible to have quotes within a string. The single quotation mark
on the keypunch is unique, imposing the restriction that a string in quotes 
can contain no other quotation marks.

The definition of number has been radically changed.
The reason for this is to cut down on the space required for the machine
subroutine which recognizes numbers.
A number is considered to be a string of digits which may include imbedded
periods, but may not begin or end with a period; moreover, periods may not
be adjacent. The use of the subscript 10 has been eliminated.

Now we have enough of the syntax defining features of the META II language
so that we can consider a simple example in some detail.

The example given here is a set of four syntax equations for defining a very
limited class of algebraic expressions.
The two operators, addition and multiplication, will be represented by +
and * respectively.
Multiplication takes precedence over addition; otherwise precedence is
indicated by parentheses.
Some examples are:

\begin{center}
{\tt A } \\
{\tt A + B } \\
{\tt A + B * C } \\
{\tt (A + B) * C }
\end{center}

The syntax which define this class of expressions are as follows:

\begin{center}
{\tt EX3 = .ID / '(' EX1 ')' .}, \\
{\tt EX2 = EX3 ('*' EX2 / .EMPTY) .}, \\
{\tt EX1 = EX2 ('+' EX1 / .EMPTY) .},
\end{center}

EX is an abbreviation for expression.
The last equation, which defines an expression of order 1, is considered
the main equation.
The equations are read in this manner.
An expression of order three is defined as an identifier or an open
parenthesis followed by an expression of order 1 followed by a closed
parenthesis.
An expression of order 2 is defined as an expression of order 3 which
may be followed by a star which is followed by an expression of order 2.
An expression of order 1 is defined as an expression of order 2,
which may be followed by a plus which is followed by an expression of order 1.

Although sequences can be defined recursively, it is more convenient and
efficient to have a special operator for this purpose.
For example, we can define a sequence of the letter A as follows:

\begin{center}
{\tt SEQA = \$ 'A' .},
\end{center}

The equations given previously are rewritten using the sequence operator
as follows:

\begin{center}
{\tt EX3 = .ID / '(' EX1 ')' .}, \\
{\tt EX2 = EX3 \$ ('*' EX3) .}, \\
{\tt EX1 = EX2 \$ ('+' EX2) .},
\end{center}

\section{Output}

Up to this point we have considered the notation in META II which describes
object language syntax. To produce a compiler, output commands are inserted
into the syntax equations.
Output from a compiler written in META II is always in an assembly language,
but not in the assembly language for the 1401.
It is for an interpreter, such as the interpreter I call the META II machine,
which is used for all compilers or the interpreters I call the VALGOL I and
VALGOL II machines, which obviously are used with their respective source
languages.
Each machine requires its own assembler, but the main difference between the
assemblers is the operation code table.
Constant codes and declarations may also be different.
These assemblers all have the same format, which is shown below.

{\tt
\begin{verbatim}
LABEL CODE ADDRESS
1- -6 8- -10 12- -70
\end{verbatim}
}

An assembly language record contains either a label or an op code of up to
3 characters, but never both.
A label begins in column 1 and may extend as far as column 70.
If a record contains an op code, then column 1 must be blank.
Thus labels may be of any length and are not attached to instructions, but
occur between instructions.

To produce output beginning in the op code field, we write .OUT and then
surround the information to be reproduced with parentheses.
A string is used for literal output and an asterisk to output the special
symbol just found in the input. This is illustrated as follows:

\begin{center}
{\tt EX3 = .ID .OUT('LD ' *) / '(' EX1 ')' .}, \\
{\tt EX2 = EX3 \$ ('*' EX3 .OUT('MLT')) .}, \\
{\tt EX1 = EX2 \$ ('+' EX2 .OUT('ADD')) .},
\end{center}

To cause output in the label field we write .LABEL followed by the item to
be output. 
For example, if we want to test for an identifier and output it in the label
field we write:

\begin{center}
{\tt .ID .LABEL *}
\end{center}

The META II compiler can generate labels of the form A01, A02, A03, \ldots
A99, B01, \ldots .
To cause such a label to be generated, one uses *1 or *2. This same label
is output whenever *1 is referred to within that execution of the equation.
The symbol *2 works in the same way.
Thus a maximum of two different labels may be generated for each execution
of any equation.
Repeated executions, whether recursive or externally initiated, result
in a continued sequence of generated labels.
Thus all syntax equations contribute to the one sequence.
A typical example in which labels are generated for branch commands
is now given.

\begin{center}
{\tt IFSTATEMENT = '.IF' EXP '.THEN' .OUT('BFP' *1) STATEMENT}
{\tt      '.ELSE' .OUT('B ' *2) .LABEL *1}
{\tt      STATEMENT .LABEL *2 .},
\end{center}

The op codes BFP and B are orders of the VALGOL I machine, and stand for
``branch false and pop'' and ``branch'' respectively.
The equation also contains references to two other equations which
are not explicitly given, viz., EXP and STATEMENT.

\section{VALGOL I - A Simple Compiler Written in META II}

Now we are ready for an example of a compiler written in META II.
VALGOL I is an extremely simple language, based on ALGOL 60, which has been
designed to illustrate the META II compiler.

The basic information about VALGOL I is given in figure 1 (the VALGOL I
compiler written in META II) and figure 2 (order list of the VALGOL I machine).
A sample program is given in figure 3.
After each line of the program, the VALGOL I commands which the compiler 
produces from that line are shown, as well as the absolute interpretive
language produced by the assembler.
Figure 4 is the output from the sample program.
Let us study the compiler written in META II (figure 1) in more detail.

The identifier PROGRAM on the first line indicates that this is the main
equation, and that control goes there first.
The equation for PRIMARY is similar to that of EX3 in our previous example,
but here numbers are recognized and reproduced with a ``load literal'' command.
TERM is what was previously EX2; and EXP1 what was previously EX1 except for
recognizing minus for subtraction.
The equation EXP defines the relational operator ``equal'', which produces a
value of 0 or 1 by making a comparison.
Notice that this is handled just like the arithmetic operators but with a
lower precedence. The conditional branch commands, ``branch true and pop'' and
``branch false and pop'', which are produced by the equations defining UNTILST
and CONDITIONALST respectively, will test the top item in the stack and branch
accordingly.

The ``assignment statement'' defined by the equation for ASSIGNST is reversed
from the convention in ALGOL 60, i.e., the location into which the computed 
value is to be stored is on the right.
Notice also that the equal sign is used for the assignment statement and that
period equal (.=) is used for the relation discussed above.
This is because assignment statements are more numerous in typical programs than
equal compares, and so the simpler representation is chosen for the more
frequently occurring.

The omission of statement labels from the VALGOL I and VALGOL II seems strange
to most programmers. This is not done because of any difficulty in their
implementation, but because of a dislike for statement labels on the part of
the author. I have programmed for several years without using a single label,
so I know that they are superfluous froma practical, as well as from a
theoretical, standpoint.
Nevertheless, it would be too much of a digression to try to justify this point
here.
The ``until statement'' has been added to facilitate writing loops without labels.

The ``conditional'' statement is similar to the one in ALGOL 60, but here the
``else'' clause is required.

The equation for ``input/output'', IOST, involves two commands, ``edit'' and 
``print''.
The words EDIT and PRINT do not begin with periods so that they will look like
subroutines written in code.
``EDIT'' copies the given string into the print area, with the first character
in the print position which is computed from the given expression.
``PRINT'' will print the current contents of the print area and then clear it
to blanks.
Giving a print command without previous edit commands results in writing a blank
line.

IDSEQ1 and IDSEQ are given to simplify the syntax equation for DEC (declaration).
Notice in the definition of DEC that a branch is given around the data.

From the definition of BLOCK it can be seen that what is considered a compound
statement in ALGOL 60 is, in VALGOL I, a special case of a block which has no
declaration.

In the definition of statement, the test for an IOST precedes that for an
ASSIGNST.
This is necessary, because if this were not done the words PRINT and EDIT
would be mistaken as identifiers and the compiler would try to translate
``input/output'' statements as if they were ``assignment'' statements.

Notice that a PROGRAM is a block and that a standard set of commands is output
after each program.
The ``halt'' command causes the machine to stop on reaching the end of the
outermost block which is the program.
The operation code SP is generated after the ``halt'' command.
This is a completely 1401-oriented code, which serves to set a word mark
at the end of the program.
It would not be used if VALGOL I were implemented on a fixed word-length machine.

\section{How the META II Compiler Was Written}

Now we come to the most interesting part of this project, and consider how the
META II compiler was written in its own language.
The interpreter called the META II machine is not a much longer 1401 program
than the VALGOL I machine.
The syntax equations for META II (figure 5) are fewer in number than those
for the VALGOL I machine (figure 1).

The META II compiler, which is an interpretive program for the META II
machine, takes the syntax equations given in figure 5 and produces an
assembly language version of this same interpretive program.
Of course, to get this started, I had to write the first compiler-writing
compiler by hand.
After the program was running, it could produce the same program as written
by hand.
Someone always asks if the compiler really produced exactly the program I
had written by hand and I have to say that it was ``almost'' the same
program.
I followed the syntax equations and tried to write just what the compiler
was going to produce.
Unfortunately I forgot one of the redundant instructions, so the results
were not quite the same.
Of couse, when the first machine-produced compiler compiled itself
the second time, it reproduced itself exactly.

The compiler originally written by hand was for a language called META I.
This was used to implement the improved compiler for META II.
Sometimes, when I wanted to change the metalanguage, I could not describe
the new metalanguage directly in the current metalanguage.
Then an intermediate language was created -- one which could be described in
the current language and in which the new language could be described.
I thought that it might sometimes be necessary to modify the assembly language
output, but it seems that it is always possible to avoid this with the
intermediate language.

The order list of the META II machine is given in figure 6.

All subroutines in META II programs are recursive.
When the program enters a subroutine a stack is pushed down by three cells.
One cell is for the exit address and the other two are for labels which may be
generated during the execution of the subroutine.
There is a switch which may be set or reset by the instructions which refer
to the input string, and this is the switch referred to by the conditional
branch commands.

The first thing in any META II machine program is the address of the first
instruction.
During the initialization of the interpreter, this address is placed into
the instruction counter.

\section{VALGOL II Written in META II}

VALGOL II is an expansion of VALGOL I, and serves as an illustration of a
fairly elaborate programming language implemented in the META II system.
There are several features in the VALGOL II machine which were not present in
the VALGOL I machine, and which require some explanation.
In the VALGOL II machine, addresses as well as numbers are put in the stack.
They are marked appropriately so that they can be distinguished at execution
time.

The main reason that addresses are allowed in the stack is that, in the case
of a subscripted variable, an address is the result of a computation.
In an assignment statement each left member is compiled into a sequence of
code which leaves an address on top of the stack.
This is done for simple variables as well as subscripted variables because the
philosophy of this compiler writing system has been to compile everything in
the most general way.
A variable, simple or subscripted, is always compiled into a sequence of
instruction which leaves an address on top of the stack.
This address is not replaced by its contents until the actual value of the
expression is needed, as in an arithmetic expression.

A formal parameter of a procedure is stored either as an address or as a value
which is computed when the procedure is called.
It is up to the load command to go through any number of indirect address in
order to place the address of a number onto the stack.
An argument of a procedure is always an algebraic expression.
In case of this expression is a variable, the value of the formal parameter
will be an address computed upon entering the procedure; otherwise,
the value of the formal parameter will be a number computed upon entering the 
procedure.

The operation of the load command is now described.
It causes the given address to be put on top of the stack.
If the content of this top item happens to be another address, then it is
replaced by that other address.
This continues until the top item on the stack is the address of something
which is not an address.
This allows for formal parameters to refer to other formal parameters to any
depth.

No distinction is made between integer and real numbers.
An integer is just a real number whose digits right of the decimal number are
zero.
Variables initially have a value called ``undefined'', and any attempt to
use this value will be indicated as an error.

An assignment statement consists of any number of left parts followed by a 
right part.
For each left part there is compiled a sequence of commands which puts an
address on top of the stack.
The right part is compiled into a sequence of instructions which leaves on top
of the stack either a number or the address of a number.
Following the instruction for the right part there is a sequence of store
commands, one for each left part.
The first command of this sequence is ``save and store'', and the rest are 
``plain'' store commands.
The ``save and store'' puts the number which is on top of the stack (or which
is referred to by the address on top of the stack) into a register
called SAVE. It then stores the contents of SAVE in the address which is
held in the next to top position of the stack.
Finally it pops the stop two items, which it has used, out of the stack.
The number, however, remains in SAVE for use by the following store commands.
Most assignment statements have only one left part, so ``plain'' store
commands are seldom produced, with the result that the number put in SAVE is
seldom used again.

The method for calling a procedure can be expained by reference to
illustrations 1 and 2.
The arguments which are in the stack are moved to their place at the top of
the procedure.

\begin{center}
{\tt
\begin{verbatim}
XXXXXXXX Function

XXXXXXXX Arguments
XXXXXXXX
........
XXXXXXXX

b        Word of one blank char-
         acter to mark the end
         of the arguments.

........ Body. Branch commands
........ cause control to go
........ around data stored in
         this area. Ends with
R        a "return" command.
\end{verbatim}
}
Illustration 1 \\
Storage Map for VALGOL II Procedures
\end{center}

\begin{center}
{\tt
\begin{verbatim}
XXXXXXXX Arguments in reverse order
XXXXXXXX
........
XXXXXXXX
     XXX Flag
     XXX Address of       Exit              XXX
........    procedure                  ........
........                               ........
Stack before executing    Stack after executing
the  call  instruction     the call instruction
\end{verbatim}
}
Illustration 2 \\
Map of the Stack Relating to Procedure Calls
\end{center}

If the number of arguments in the stack does not correspond to the number of
arguments in the procedure, an error is indicated.
The ``flag'' in the stack works like this.
In the VALGOL II machine there is a flag register.
To set a flag in the stack, the contents of this register is put on top of
the stack, then the address of the word above the top of the stack is put into
the flag register.
Initially, and whenever there are no flags in the stack, the flag register
contains blanks.
At other times it contains the address of the word in the stack which is just
above the uppermost flag.
Just before a call instruction is executed, the flag register contains the
address of the word in the stack which is two above the word containing the
address of the procedure to be executed.
The call instruction picks up the arguments from the stack, beginning with
the one stored just above the flag, and continuing to the top of the stack.
Arguments are moved into the appropriate places at the top of the procedure
being called.
An error message is given if the number of arguments in the stack does not
correspond to the number of places in the procedure.
Finally the old flag address, which is just below the procedure address in
the stack, is put in the flag register.
The exit address replaces the address of the procedure in the stack, and all
the arguments, as well as the flag, are popped out.
There are just two op codes which affect the flag register.
The code ``load flag'' puts a flag into the stack, and the code ``call'' takes
one out.

The library function ``whole'' truncates a real number.
It does not convert a real number to an integer, because no distinction is made
between them.
It is substituted for the recommended function ``ENTIER'' primarily because
truncation takes fewer machine instructions to implement.
Also, truncation seems to be used more frequently.
The procedure ENTIER can be defined in VALGOL II as follows:

\begin{center}
{\tt .PROCEDURE ENTIER(X) .}, \\
{\tt \ \ \ .IF O .L= X .THEN WHOLE (x) .ELSE} \\
{\tt \ \ \ .IF WHOLE(X) = X .THEN X .ELSE} \\
{\tt \ \ \ WHOLE(X) -1}
\end{center}

The ``for statement'' in VALGOL II is not the same as it is in ALGOL.
Exactly one list element is required.
The ``step .. until'' portion of the element is mandatory, but the ``while''
portion may be added to terminate the loop immediately upon some condition.
The iteration continues so long as the value of the variable is less than
or equal to the maximum, irrespective of the sign of the incrment.
The iteration continues so long as the value of the variable is less than
or equal to the maximum, irrespective of the sign of the increment.
Illustration 3 is an example of a typical ``for statement''.
A flow chart of this statement is given in illustration 4.

\begin{center}
{\tt
\begin{verbatim}
.FOR I = 0 .STEP 1 .UNTIL N .DO
       <statement>
       SET         Set switch to indicate first
A91                time through
       LD    I
       FLP         Test for first time through
       BFP   A92
       LDL   0
       SST         Initialize variable.
       B     A93
A92
       LDL   1     Increment variable.
       ADS
A93
       RSR         Compare variable to maximum
       LD    N
       LEQ
       BFP   A94
       <statement>
       RST         Reset switch to indicate not
                   first time through
       B     A91
A94
\end{verbatim}
}
Illustration 3 \\
Compilation of a typical ``for statement'' \\
in VALGOL II
\end{center}

\begin{center}
{\tt
\begin{verbatim}
TODO GRAPHICS
\end{verbatim}
}
Illustration 4 \\
Flow chart of the ``for statement''
given in figure 12
\end{center}

Figure 7 is a listing of the VALGOL II compiler written in META II.
Figure 8 gives the order list of the VALGOL II machine.
A sample program to take a determinant is given in figure 9.

\section{Backup vs. No Backup}

Suppose that, upon entry to a recursive subroutine, which represents some
syntax equation, the position of the input and output are saved.
When some non-first term of a component is not found, the compiler does not
have to stop with an indication of a syntax error.
It can back-up the input and output and return false.
The advantages of backup are as follows:

\begin{enumerate}
\item It is possible to describe languages, using backup, which cannot be
described without backup.
\item Even for a language which can be described without backup, the syntax
equations can often be simplified when backup is allowed.
\end{enumerate}

The advantages claimed for no-backup are as follows:

\begin{enumerate}
\item Syntax analysis is faster.
\item It is possible to tell whether syntax equations will work just by
examining them, without following through numerous examples.
\end{enumerate}

The fact that rather sophisticated languages such as ALGOL and COBOL can be
implemented without backup is pointed out by various people, including 
Conway, and they are aware of the speed advantages of so doing.
I have seen no mention of the second advantage of no-backup, so I will explain
this in more detail.

Basically one writes alternations in which each term begins with a different
symbol.
Then it is not possible for the compiler to go down the wrong path.
This is made more compicated because of the use of ``.EMPTY''.
An optional item can never be followed by something that begins with
the same symbol it begins with.

The method describe above is not the only way in which backup can be handled.
Variations are worth considering, as a way may be found to have the advantages
of both backup and no-backup.

\section{Further Development of META Languages}

As mentioned earlier, META II is not presented as a standard language, but
as a point of departure from which a user may develop his own META language.
The term ``META language,'' with ``META'' in capital letters, is used to
denote any compiler-writing language so developed.

The language which Schmidt implemented on the PDP-1 was based on META I.
He has now implemented an improved version of this language for a Beckman
machine.

Rutman has implemented LOGIK, a compiler for bit-time simulation, on the 7090.
He uses a META language to compile Boolean expressions into efficient machine
code.
Schneider and Johnson have implemented META 3 on the IBM 7094, with the goal of
producing an ALGOL compiler which generates efficient machine code.
They are planning a META language which will be suitable for any block
structured language.
To this compiler-writing language they give the name META 4 (pronounced metaphor).

\section{References}

TODO: real bibtex and citations

\begin{enumerate}
\item Schmidt, L. ``Implementation of a Symbol Manipulator for Heuristic Translation''
\item Metcalfe, Howard, ``A Parameterized Compiler Based on Mechanical Linguistics''
\item Schorre, Val, ``A Syntax-Directed SMALGOL for the 1401''
\item Glennie, A., ``On the Syntax Machine and the Construction of a Universal Compiler,''
\item Conway, Melvin E., ``Design of a Separable Transition-Diagram Compiler''
\item Irons, E. T., ``The Structure and Use of the Syntax-Directed Compiler''
\item Bastian, Lewis, ``A Phrase-Structure Language Translator''
\item Chomsky, Noam, ``Syntax Structures''
\item Rutman, Roger, ``LOGIK, A Syntax Directed Compiler for Computer Bit-Time Simulation''
\item Schneider, F. W., and G. D. Johnson, ``A Syntax-Directed Compiler-Writing Compiler to Generate Efficient Code''
\end{enumerate}

\pagebreak

\section{THE VALGOL I COMPILER WRITTEN IN META II LANGUAGE\\FIGURE 1}
\verbatiminput{valgol_1_in_meta_2.txt}

\section{ORDER LIST OF THE VALGOL I MACHINE\\FIGURE 2}

\subsection{Machine Codes}
\begin{description}
\item[LD AAA]      Load -- Put the contents of the address AAA on top of the stack.

\item[LDL NUMBER]  Load Literal -- Put the given number on top of the stack.

\item[ST AAA]      Store -- Store the number which is on top of the stack into the address AAA and pop up the stack.

\item[ADD]         Add -- Replace the two numbers which are on top of the stack with their sum.

\item[SUB]         Subtract -- Subtract the number which is on top of the stack from
the number which is next to the top, then replace them by this difference.

\item[MLT]         Multiply -- Replace the two numbers which are on top of the stack
with their product

\item[EQU]         Equal -- Compare the two numbers on top of the stack. Replace them
by the integer 1, if they are equal, or by the integer 0, if they are unequal.

\item[B AAA]       Branch -- Branch to the address AAA.

\item[BFP AAA]     Branch false and pop -- Branch to the address AAA if the top term
in the stack is the integer 0. Otherwise, continue in sequence. In either case,
pop up the stack.

\item[BTP AAA]     Branch true and pop -- Branch to the address AAA if the top term
in the stack is not the integer 0. Otherwise, continue in sequence. In either
case, pop up the stack.

\item[EDT STRING]  Edit -- Round the number which is on top of the stack to the nearest
integer N. Move the given string into the print area so that its first character
falls on print position N. In case this would cause characters to fall outside
the print area, no movement takes place.

\item[PNT]        Print -- Print a line, then space and clear the print area.

\item[HLT]        Halt -- Halt.
\end{description}
\subsection{Constant and Control Codes}
\begin{description}
\item[SP  N]      Space -- N = 1--9. Constant code producing N blank spaces.

\item[BLK NNN]    Block -- Produces a block of NNN eight character words.

\item[END]        End -- Denotes the end of the program.
\end{description}

\pagebreak

\section{THE META II COMPILER WRITTEN IN ITS OWN LANGUAGE\\FIGURE 5}
\verbatiminput{meta_2_in_meta_2.txt}

\section{ORDER LIST OF THE META II MACHINE\\FIGURE 6}

\subsection{Machine Codes}
\begin{description}
\item[TST STRING] Test: After deleting initial blanks in the input string, compare it to the string given as argument. If the comparison is met, delete the matched portion from the input and set switch. If not met, reset switch.
\item[ID] Identifier: After deleting initial blanks in the input string, test if it begins with an identifier, ie., a a letter followed by a sequence of letters and/or digits. If so, delete the identifier and set switch. If not, reset switch.
\item[NUM] Number: After deleting initial blanks in the input string, test if it begins with a number. A number is a string of digits which may contain imbeded periods, but may not begin or end with a period.  Moreover, no two periods may be next to one another. If a number is found, delete it and set switch. If not, reset switch.
\item[SR] String: After deleting initial blanks in the input string, test if
it begins with a string. Ie., a single quote followed by a sequence of any
characters other than single quote followed by another single quote.
If a string is found, delete it and set switch.
If not, reset switch.
\item[CLL AAA] Call: Enter the subroutine beginning in location AAA.
If the top two terms of the stack are blank, push the stack down by one cell.
Otherwise, push it down by three cells.
Set a flag in the stack to indicate whether it has been pushed by one
or three cells.
This flag and the exit address go into the third cell.
Clear the top two cells to blanks to indicate that they can accept addresses
which may be generated within the subroutine.
\item[R] Return: Return to the exit address, popping up the stack by one
or three cells according to the flag.
If the stack is popped by only one cell, then clear the top two cells to
blanks, beause they wer blank when the subroutine was entered.
\item[SET] Set: Set branch switch on.
\item[B   AAA] Branch: Branch unconditionally to location AAA.
\item[BT  AAA] Branch if true: Branch to location AAA if switch is on.
Otherwise, continue in sequence.
\item[BF  AAA] Branch if false: Branch to location AAA if switch is off.
Otherwise, continue in sequence.
\item[BE] Branch to error if false: Halt if switch is off, otherwsie,
continue in sequence.
\item[CL  STRING] Copy literal: Output the variable length string given as the argument.
A blank character will be inserted in the output following the string.
\item[CI] Copy input: Output the last sequence of characters deleted from the
input string. This command may not function properly if the last command
which could cause deletion failed to do so.
\item[GN1] Generate 1: This concerns the current label 1 cell, ie., the next
to top cell in the stack, which is either clear or contains a generated label.
If clear, generate a label and put it into that cell.
Whether the label has just been put into the cell or was already there, output
it.
Finally, insert a blank character in the output following the label.
\item[GN2] Generate 2: Same as GN1, except that it concerns the current label
2 cell ie., the top cell in the stack.
\item[LB] Label: Set the output counter to card column 1.
\item[OUT] Output: Punch card and reset output counter to card column 8.
\end{description}

\subsection{Constant and control codes}
\begin{description}
\item[ADR IDENT] Address: Produces the address which is assigned to the given
identifier as a constant.
\item[END] End: Denotes the end of the program.
\end{description}

\pagebreak

\section{VALGOL II COMPILER WRITTEN IN META II\\FIGURE 7}
\verbatiminput{valgol_2_in_meta_2.txt}


\pagebreak

\section{ORDER LIST OF THE VALGOL II MACHINE\\FIGURE 8}

\subsection{Machine Codes}
\begin{description}
\item[LD  AAA] Load: Put the address AAA on top of the stack.
\item[LDL NUMBER] Load literal: Put the given number on top of the stack.
\item[SET] Set: Put the integer 1 on top of the stack.
\item[RST] Reset: Put the integer 0 on top of the stack.
\item[ST] Store: Store the contents of the register, Stack1, in the address
which is on top of the stack, then pop up the stack.
\item[ADS] Add To Storage: Add the number on top of the stack to the number
whose address is next to the top, and place the sum in the register, stack1.
Then store the contents of that register in that address, and pop the top
two items out of the stack.
\item[SST] Save And Store: Put the number on top of the stack into the register,
stack1. Then store the contents of that register in the address which is the
next to top term of the stack.
The top two items are popped out of the stack.
\item[RSR] Restore: Put the contents of the register, stack1, on top of the stack.
\item[ADD] Add: Replace the two numbers which are on top of the stack with their sum.
\item[SUB] Subtract: Subtract the number which is on top of the stack from
the number which is next to the top, then replace them by this difference.
\item[MLT] Multiply: Replace the two numbers which are on top of the stack
with their product.
\item[DIV] Divide: Divide the number which is next to the top of the stack by the
nmber which is on top of the stack, then replace them by this quotient.
\item[NEG] Negate: Change the sign of the number on top of the stack.
\item[WHL] Whole: Truncate the number which is on top of the stack.
\item[NOT] Not: If the top term in the stack is the integer 0, then replace
it with the integer 1. Otherwise, replace it with the integer 0.
\item[LEQ] Less than or equal: If the number which is next to the top of the
stack is less than or equal to the number on top of the stack, then replace
them with the integer 1. Otherwise, replace them with the integer 0.
\item[LES] Less than: If the number which is next to the top of the stack is
less than the number on top of the stack, then replace them with the integer
1. Otherwise, replace them with the integer 0.
\item[EQU] Equal: Compare the two numbers on top of the stacl.
Replace them by the integer 1, if they are equal, or by the integer 0, if 
they are unequal.
\item[B   AAA] Branch: Branch to the address AAA.
\item[BT  AAA] Branch True: Branch to the address AAA if the top term in
the stack is not the integer 0. Otherwise, continue in sequence.
Do not pop up the stack.
\item[BF  AAA] Branch False: Branch to the address AAA If the top term in
the stack is the integer 0. Otherwise, continue in sequence.
Do not pop up the stack.
\item[BTP AAA] Branch True and Pop: Branch to the address AAA if the top
term in the stack is not the integer 0. Otherwise, continue in sequence.
In either case, pop up the stack.
\item[BFP AAA] Branch False and Pop: Branch to the address AAA if the
top term in the stack is the integer 0.
Otherwise, continue in sequence.
In either case, pop up the stack.
\item[CLL] Call: Enter a procedure at the address which is below the flag.
\item[LDF] Load Flag: Put the address which is in the flag register on top
of the stack, and put the address of the top of the stack into the flag register.
\item[R   AAA] Return: Return from procedure.
\item[AIA] Array Increment Address: Increment the address which is next
to the top of the stack by the integer which is on top of the stack, and
replace these by the resulting address.
\item[FLP] Flip: Interchange the top two terms of the stack.
\item[POP] Pop: Pop up the stack.
\item[EDT STRING] Edit: Round the number which is on top of the stack to the
nearest integer N. Move the given string into the print area so that its first
character falls on print position N.
In case this would cause characters to fall outside the print area, no movement
takes place.
\item[PNT] Print: Print a line, then space and clear the print area.
\item[EJT] Eject: Position the paper in the printer to the top line of the next page.
\item[RED] Read: Read the first N numbers from a card and store them beginning
in the address which is next to the top of the stack.
The integer N is the top term of the stck.
Pop out both the address and the integer.
Cards are punched with up to 10 eight digit numbers.
Decimal point is assumed to be in the middle.
An 11-punch over the rightmost digit indicates a negative number.
\item[WRT] Write: Print a line of N numbers beginning in the address which
is next to the top of the stack.
The integer N is the top term of the stack.
Pop out both the address and the integer.
Twelve character positions are allowed for each number.
There are four digits before and four digits after the decimal.
Leading zeroes in front of the decimal are changed to blanks.
If the number is negative, a minus sign is printed in the position before
the first non-blank character.
\item[HLT] Halt: Halt.
\end{description}
\subsection{Constant and control codes}
\begin{description}
\item[SP  N] Space: N = 1--9. Constant code producing N blank spaces.
\item[BLK NNN] Block: Produces a block of NNN eight character words.
\item[END] End: Denotes the end of the program.
\end{description}

TODO: mark the opcodes that say 'see note 1' and 'see note 2'

Note 1: If the top item in the stack is an address, it is replaced by its
contents before beginning this operation.

Note 2: Same as note 1, but applies to the top two items.

\section{EXAMPLE PROGRAM IN VALGOL II\\FIGURE 9}
\verbatiminput{example_in_valgol_2.txt}


\end{document}
